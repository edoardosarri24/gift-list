\chapter{Introduction}
This document outlines the concept for a "Gift List" application, designed to streamline the process of gift-giving for various celebrations and events. The core idea is to provide a platform where a celebrant (e.g., a birthday person, a couple getting married) can curate and share a personalized wish list of gifts they would like to receive.

The application serves two primary user roles:
\begin{itemize}
    \item The Celebrant \\
        This user creates and manages their gift list. They can add, remove, and describe desired items. Critically, the celebrant will not have visibility into the real-time status of their gift list regarding which items have been claimed. This maintains the element of surprise for them.
    \item The Guest \\
        Guests receive a link to the celebrant's gift list. They can browse the available items and, crucially, "claim" a gift they intend to purchase. Once a gift is claimed by one guest, it becomes visibly unavailable to all other guests. This mechanism prevents duplicate gifts and helps guests coordinate their purchases efficiently.
\end{itemize}

The primary goal of this application is to simplify gift coordination, reduce redundant gifts, and enhance the gift-giving experience by ensuring celebrants receive desired items while preserving the element of surprise.

\begin{figure}[htbp]
    \centering
    \includegraphics[width=\textwidth]{images/1-introduction/use-case.pdf}
    \caption{Use case diagram of Gift List requirements.}
\end{figure}

%%%%%%%%%%%%%%%%%%%%%%%%%%%%%%%%%%%%%%%%%%%%%%%%%%%%%%%%
\section{Requirements}
In this Section we will define the requirements for the application. The application must meet the following functional requirements:
\begin{itemize}
    \item \textbf{User Authentication (Celebrant Only):} The application must allow Celebrants to register and log-in to manage their lists securely. Guests should be able to access lists via a shared link without requiring an account, to minimize friction.
    \item \textbf{List Management:} Authenticated Celebrants must be able to create and delete gift lists.
    \item \textbf{Item Management:} Celebrants must be able to add and remove items from their lists. For each item, they can provide a name, description, and an optional external URL (e.g., to an online store).
    \item \textbf{List Access:} Guests must be able to view a specific gift list using a unique URL provided by the Celebrant.
    \item \textbf{Item Claiming:} Guests must be able to mark an item as "claimed" or "purchased". This action must be visible to other Guests to prevent duplicate purchases but must remain hidden from the Celebrant to preserve the surprise.
    \item \textbf{Claimed Item Removal:} If a Celebrant removes an item that has already been claimed by a Guest, the system must allow the removal but must immediately notify the Guest who claimed it. This prevents the Guest from purchasing an item that is no longer desired, while maintaining the Guest's anonymity.
\end{itemize}

%%%%%%%%%%%%%%%%%%%%%%%%%%%%%%%%%%%%%%%%%%%%%%%%%%%%%%%%
\section{Documentation Structure}
The document is structured as follows:
\begin{itemize}
    \item Chapter~\ref{chap:backend} details the backend architecture and logic.
    \item Chapter~\ref{chap:frontend} describes the frontend user interface and experience.
    \item Chapter~\ref{chap:deployment} outlines the deployment process and configuration.
\end{itemize}