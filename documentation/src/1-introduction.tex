\chapter{Introduction}
This document outlines the concept for a "Gift List" application, designed to streamline the process of gift-giving for various celebrations and events. The core idea is to provide a platform where a celebrant (e.g., a birthday person, a couple getting married) can curate and share a personalized wish list of gifts they would like to receive.

The application serves two primary user roles:
\begin{itemize}
    \item The Celebrant \\
        This user creates and manages their gift list. They can add, remove, and describe desired items. Critically, the celebrant will not have visibility into the real-time status of their gift list regarding which items have been claimed. This maintains the element of surprise for them.
    \item The Guest \\
        Guests receive a link to the celebrant's gift list. They can browse the available items and, crucially, "claim" a gift they intend to purchase. Once a gift is claimed by one guest, it becomes visibly unavailable to all other guests. This mechanism prevents duplicate gifts and helps guests coordinate their purchases efficiently.
\end{itemize}

The primary goal of this application is to simplify gift coordination, reduce redundant gifts, and enhance the gift-giving experience by ensuring celebrants receive desired items while preserving the element of surprise.

%%%%%%%%%%%%%%%%%%%%%%%%%%%%%%%%%%%%%%%%%%%%%%%%%%%%%%%%
\section{Requirements}

%%%%%%%%%%%%%%%%%%%%%%%%%%%%%%%%%%%%%%%%%%%%%%%%%%%%%%%%
\section{Documentation Structure}
The document is structured as follows:
\begin{itemize}
    \item Chapter~\ref{chap:backend} details the backend architecture and logic.
    \item Chapter~\ref{chap:frontend} describes the frontend user interface and experience.
    \item Chapter~\ref{chap:deployment} outlines the deployment process and configuration.
\end{itemize}